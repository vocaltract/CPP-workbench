\documentclass[UTF8]{ctexart}
%\documentclass[a4paper]{article}
% \usepackage[margin=1.25in]{geometry}
\usepackage[inner=2.0cm,outer=2.0cm,top=2.5cm,bottom=2.5cm]{geometry}
%\usepackage{CJK}
\usepackage{color}
\usepackage{graphicx}
\usepackage{amssymb}
\usepackage{amsmath}
\usepackage{amsthm}
\usepackage{bm}
\usepackage{hyperref}
\usepackage{multirow}
\usepackage{enumerate}
\usepackage{listings}
\usepackage{xcolor}
\newcommand{\homework}[5]{
    \pagestyle{myheadings}
    \thispagestyle{plain}
    \newpage
    \setcounter{page}{1}
    \noindent
    \begin{center}
    \framebox{
        \vbox{\vspace{2mm}
        \hbox to 6.28in { {\bf 高级程序设计 \hfill #2} }
        \vspace{6mm}
        \hbox to 6.28in { {\Large \hfill #1 \hfill} }
        \vspace{6mm}
        \hbox to 6.28in { {\it 指导老师: {\rm #3} \hfill 姓名: {\rm #4},学号: {\rm #5}}}
        \vspace{2mm}}
    }
    \end{center}
    % \markboth{#4 -- #1}{#4 -- #1}
    \vspace*{4mm}
}


\newenvironment{solution}
{\color{blue} \paragraph{Solution.}}
{\newline \qed}

\begin{document}



%==========================Put your name and id here==========================
\homework{实验报告}{2020 春}{陈家骏\phantom{ }黄书剑}{王晨渊}{181220057}

%\paragraph{Notice}
%\begin{itemize}
%    \item The submission email is: \textbf{njuoptfall2019@163.com}.
%    \item Please use the provided \LaTeX{} file as a template. If you are not familiar with \LaTeX{}, you can also use Word to generate a \textbf{PDF} file.
%\end{itemize}
\lstset{numbers=left,numberstyle=\tiny,keywordstyle=\color{blue!70},commentstyle=\color{red!50!green!50!blue!50},frame=shadowbox, rulesepcolor=\color{red!20!green!20!blue!20},escapeinside=``,xleftmargin=2em,xrightmargin=2em, aboveskip=1em}



\paragraph{}
~\\
\section{说明}
    本人在代码中使用了大量的预处理命令。宏LOGGING决定是否启用Log()将提示输出到stdout。在数据抽象中,宏LINKED\_NODE和宏ARRAY决定使用单链表或循环数组作为实现方式。
\section{过程抽象和封装}
\begin{lstlisting}[language={[ANSI]C++}]
    #include <cstdio>
    using namespace std;
    typedef int QUEVAL;
    
    
    #define LOGGING
    #ifdef LOGGING
    #define Log(format, ...)\
        printf("\33[1;35m[%s,%d,%s] " format "\33[0m\n",\
            __FILE__, __LINE__, __func__, ## __VA_ARGS__)
    #else
    #define Log(format, ...)  
    #endif
    
    bool is_empty();
    bool is_full();
    int insert(QUEVAL num);
    int del(QUEVAL &num);
    
    
    const int QUEUE_SZ = 3;
    int front = 0, rear = 0;
    bool empty=true, full=false;
    QUEVAL queue[QUEUE_SZ];
    
    
    int main()
    {
        QUEVAL num;
        del(num);
        insert(1);
        insert(1);
        insert(1);
        insert(2);
        del(num);
        del(num);
        del(num);
        del(num);
    }
    
    
    bool is_empty()
    {
        return empty;
    }
    
    bool is_full()
    {
        return full;
    }
            
    int insert(QUEVAL num)
    {
        if(full)
        {
            Log("The queue is full!Insertion failed!");
            return -1;
        }
        else
        {
            empty = false;
            queue[rear] = num;
            rear = (rear+1)%QUEUE_SZ;
            if(rear == front) full = true;
            return 0;
        }
    }
    int del(QUEVAL &num)
    {
        if(empty)
        {
            Log("The queue is empty!Deletion failed!");
            return -1;
        }
        else
        {
            full = false;
            num = queue[front];
            front = (front+1)%QUEUE_SZ;
            if(rear == front) empty = true;
            return 0;
        }
    }
\end{lstlisting}


%If $f$ is a continuous function on some interval $\mathbf{I}$,
%\begin{enumerate}[a)]
%    \item Prove that $f$ is a convex function if and only if 
%\end{enumerate}

\section{数据抽象和封装}
\begin{lstlisting}[language={[ANSI]C++}]
    #include <cstdio>
    using namespace std;
    /*使用循环数组和双向链表实现*/
    /*使用ARRAY LINKED_NODE LOGGING 三个宏进行控制*/
    //#define ARRAY
    #define LINKED_NODE 1
    #define LOGGING
    
    
    #ifdef LOGGING
    #define Log(format, ...)\
        printf("\33[1;35m[%s,%d,%s] " format "\33[0m\n",\
            __FILE__, __LINE__, __func__, ## __VA_ARGS__)
    #else
    #define Log(format, ...)  
    #endif
    #ifdef ARRAY
    const int QUEUE_SZ = 100;
    #elif LINKED_NODE
    struct Node{
        int val;
        Node* next;
    };
    #endif
    
    class Queue
    {
        public:
        #ifdef ARRAY
            Queue() 
            {
                front=rear=0;
                empty = true;
                full = false;
            }
            bool is_empty()
            {
                return empty;
            }
            bool is_full()
            {
                return full;
            }
            int insert(int num)
            {
                if(full)
                {
                    Log("The queue is full!Insertion failed!");
                    return -1;
                }
                else
                {
                    empty = false;
                    queue[rear] = num;
                    rear = (rear+1)%QUEUE_SZ;
                    if(rear == front) full = true;
                    return 0;
                }
            }
            int del(int &num)
            {
                if(empty)
                {
                    Log("The queue is empty!Deletion failed!");
                    return -1;
                }
                else
                {
                    full = false;
                    num = queue[front];
                    front = (front+1)%QUEUE_SZ;
                    if(rear == front) empty = true;
                    return 0;
                }
            }
        #elif LINKED_NODE
        Queue()
        {
            front = rear = NULL;
            count = 0;
        }
        ~Queue()
        {
            while(front)
            {
                Node* tmp = front;
                front = front->next;
                delete tmp;
            }
        }
    
        bool is_empty()
        {
            return !count;
        }
    
        int del(int &num)
        {
            if(is_empty())
            {
                Log("The queue is empty!Deletion failed!");
                return -1;           
            }
            else
            {
                Node * temp = front;
                front = front->next;
                num = temp->val;
                delete temp;
                count --;
                return 0;
            }
        }
    
        int insert(int num)
        {
            if(!is_empty())
            {
                rear->next = new Node;
                rear = rear->next;
                rear->val = num;
                rear->next = NULL;
            }
            else
            {
                rear = front = new Node;
                rear ->val = num;
                rear ->next = NULL;
            }
            count ++;
            return 0;
        }
        #endif
        private:
        #ifdef ARRAY
            int front, rear;
            int queue[QUEUE_SZ];
            bool full,empty;
        #elif LINKED_NODE
            Node *front, *rear;
            int count;
        #endif
    };
    
    int main()
    {
        Queue my_queue;
        int num;
        my_queue.del(num);
        my_queue.insert(1);
        my_queue.insert(1);
        my_queue.insert(1);
        my_queue.insert(1);
        my_queue.del(num);
        my_queue.del(num);
        my_queue.del(num);
        my_queue.del(num);
        my_queue.del(num);
    }   
\end{lstlisting}
\section{数据抽象和封装的优点}
    相比过程抽象,数据抽象将front,rear等指示循环数组头尾的重要信息封装了起来,数据与功能联系紧密,防止无关的外部函数对其进行操作,避免了可能存在的函数副作用对数据的影响。而过程抽象的实现将front与rear定义为全局变量,数据与功能联系松散,一方面让模块的划分更加模糊,另一方面也造成了潜在的在未来编写其他函数或变量时,因强符号、弱符号的区别造成意想不到的符号重定义、语义与实际链接对象不符等链接错误。


\end{document}
