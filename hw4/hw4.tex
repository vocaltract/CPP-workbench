\documentclass[UTF8]{ctexart}
%\documentclass[a4paper]{article}
% \usepackage[margin=1.25in]{geometry}
\usepackage[inner=2.0cm,outer=2.0cm,top=2.5cm,bottom=2.5cm]{geometry}
%\usepackage{CJK}
\usepackage{color}
\usepackage{graphicx}
\usepackage{amssymb}
\usepackage{amsmath}
\usepackage{amsthm}
\usepackage{bm}
\usepackage{hyperref}
\usepackage{multirow}
\usepackage{enumerate}
\usepackage{listings}
\usepackage{xcolor}
\usepackage{fontspec}
\setmainfont{Times New Roman}

\newcommand{\homework}[5]{
    \pagestyle{myheadings}
    \thispagestyle{plain}
    \newpage
    \setcounter{page}{1}
    \noindent
    \begin{center}
    \framebox{
        \vbox{\vspace{2mm}
        \hbox to 6.28in { {\bf 高级程序设计 \hfill #2} }
        \vspace{6mm}
        \hbox to 6.28in { {\Large \hfill \bf #1 \hfill} }
        \vspace{6mm}
        \hbox to 6.28in { {\it 指导老师: {\rm #3} \hfill 姓名: {\rm #4},学号: {\rm #5}}}
        \vspace{2mm}}
    }
    \end{center}
    % \markboth{#4 -- #1}{#4 -- #1}
    \vspace*{4mm}
}


\newcommand{\yahei}{\setCJKfamilyfont{yahei}{Microsoft YaHei} \CJKfamily{yahei}}




\newenvironment{solution}
{\color{blue} \paragraph{Solution.}}
{\newline \qed}

\begin{document}


%大标题在这里改!!!!!!!!!
\homework{实验报告}{2020 春}{陈家骏\phantom{ }黄书剑}{王晨渊}{181220057}

\lstset{numbers=left,numberstyle=\tiny,keywordstyle=\color{blue!70},commentstyle=\color{red!50!green!50!blue!50},frame=shadowbox, rulesepcolor=\color{red!20!green!20!blue!20},escapeinside=``,xleftmargin=2em,xrightmargin=2em, aboveskip=1em}

\section{概念题}
\subsection{C++中操作符的重载遵循哪些基本原则?}
只能重载C++语言中已有的操作符,不可臆造新的操作符

可以重载C++中除. .* ?: :: sizeof 外的所有操作符

需要遵循已有操作符的语法,一方面不能改变操作数的个数,另一方面不能改变原操作符的优先级和结合性。

尽量遵循已有操作符原来的语义。语言本身没有对此做任何规定,使用者需要自己把握。

\subsection{简述单目运算符(++,--)前置重载和后置重载的差别。}

如果不做特殊处理,前置和后置共用同一个重载函数,但失去了原先的语义。

前置为左值表达式,后置为右值表达式。

因此,定义前置的自增(自减)操作符时,函数原型为

<返回值类型> operator \#(); 其中返回值为引用类型。

定义后置的自增(自减)操作符时,函数原型为

<返回值类型> operator \#(int); 其中返回值为常量类型。




\section{编程题}
\subsection{修改原程序}
原程序不能正常运行。

由于在类中没有显式地重载=操作符,a2=a1调用的是C++为其创建的隐式默认重载函数,默认将a1的所有成员原样赋值给a2。这样a2.p和a1.p指向了相同的内存空间。当a1或a2其中一个消亡,另一个没有消亡的时候,没有消亡的对象对于p指向的内容的访问会出现问题。并且a1和a2都消亡时,
指向的空间被释放两次引发报错。

应该将其修改为以下内容。
\begin{lstlisting}[language={[ANSI]C++}]
    #include <iostream>
    using namespace std;
    class A
    {
    public:
        int x;
        int *p;
        A()
        {
            p = new int(0);
            x = 0;
        }
    
        A& operator = (const A& a)
        {
            if(&a==this) return *this;
            x = a.x;
            delete p;
            p = new int(*a.p);
            return *this;
        }
    
        A& operator =(A&& a)
        {
            if(&a == this) return *this;
            x = a.x;
            delete p;
            p = new int(*a.p);
            a.p = NULL;
            return *this;
        }
    
        A(int m, int n)
        {
            p = new int(n);
            x = m;
        }
        ~A()
        {
            delete p;
            x = 0;
        }
    
    
    };
    
    int main()
    {
        A a1(6, 8);
        A a2;
        a2 = a1;
        cout << "a1.x = " << a1.x << ", "
             << "*(a1.p) = " << *(a1.p) << endl;
        cout << "a2.x = " << a2.x << ", "
             << "*(a2.p) = " << *(a2.p) << endl;
        cout << "a1.p = " << a1.p << endl;
        cout << "a2.p = " << a2.p << endl;
        return 0;
    }
 \end{lstlisting}
 \subsection{设计一个日期类Date}
\begin{lstlisting}[language={[ANSI]C++}]
    #include <atltime.h>
    #include <iostream>
    class Date
    {
    public:
        friend Date operator+(const Date& date, int day);
        friend Date operator-(const Date& date, int day);
        Date()
        {
            cur = CTime::GetCurrentTime();
        }
        Date(const CTime& date)
        {
            cur = date;
        }
        Date(int year, int month, int day)
            :cur(year,month,day,0,0,0)
        {
        }
    
        Date& operator =(const Date& another)
        {
            if (&another == this) return *this;
            cur = another.cur;
            return *this;
        }
        Date& operator++()
        {
            cur += one_day;
            return *this;
        }
        const Date operator++(int)
        {
            cur += one_day;
            return *this;
        }
        Date& operator--()
        {
            cur -= one_day;
            return *this;
        }
        const Date operator--(int)
        {
            cur -= one_day;
            return *this;
        }
        int operator-(const Date& another)
        {
            return (cur - another.cur).GetDays();
        }
    
        void output() const
        {
            std::cout<<cur.GetYear()<<"-"<<cur.GetMonth()<<"-"<<cur.GetDay()<<"\n";
        }
    private:
        CTime cur;
        static CTimeSpan one_day;
    };
    CTimeSpan Date::one_day(1, 0, 0, 0);
    
    Date operator+(const Date& date, int day)
    {
        CTimeSpan add_days(day, 0, 0, 0);
        CTime new_cdate(date.cur);
        new_cdate += add_days;
        Date new_date(new_cdate);
        return new_date;
    }
    Date operator-(const Date& date, int day)
    {
        CTimeSpan add_days(day, 0, 0, 0);
        CTime new_cdate(date.cur);
        new_cdate -= add_days;
        Date new_date(new_cdate);
        return new_date;
    }
    
    
\end{lstlisting}
\subsection{不会越界的String类}
    \begin{lstlisting}[language={[ANSI]C++}]
        #include<iostream>
        #include<cstring>
        class String
        {
        
        public:
            String()
            {
                str=NULL;
                str_len = 0;
            }
            String(char* s)
            {
                str_len =strlen(s); 
                str = new char[str_len+1];
                strcpy(str,s);
            }
            char& operator[](int i)
            {
                return i>=str_len?str[str_len-1]:str[i];   
            }
            String& operator=(const String& s)
            {
                if(&s==this) return *this;
                delete[] str;
                str_len = s.str_len;
                str = new char[str_len+1];
                strcpy(str,s.str);
                return *this;
            }
            ~String()
            {
                if(!str_len) delete[] str;
            }
        private:
            char* str;
            int str_len;
        };
\end{lstlisting}

\end{document}
