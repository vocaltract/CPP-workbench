\documentclass[UTF8]{ctexart}
%\documentclass[a4paper]{article}
% \usepackage[margin=1.25in]{geometry}
\usepackage[inner=2.0cm,outer=2.0cm,top=2.5cm,bottom=2.5cm]{geometry}
%\usepackage{CJK}
\usepackage{color}
\usepackage{graphicx}
\usepackage{amssymb}
\usepackage{amsmath}
\usepackage{amsthm}
\usepackage{bm}
\usepackage{hyperref}
\usepackage{multirow}
\usepackage{enumerate}
\usepackage{listings}
\usepackage{xcolor}
\usepackage{fontspec}
\setmainfont{Times New Roman}

\newcommand{\homework}[5]{
    \pagestyle{myheadings}
    \thispagestyle{plain}
    \newpage
    \setcounter{page}{1}
    \noindent
    \begin{center}
    \framebox{
        \vbox{\vspace{2mm}
        \hbox to 6.28in { {\bf 高级程序设计 \hfill #2} }
        \vspace{6mm}
        \hbox to 6.28in { {\Large \hfill \bf #1 \hfill} }
        \vspace{6mm}
        \hbox to 6.28in { {\it 指导老师: {\rm #3} \hfill 姓名: {\rm #4},学号: {\rm #5}}}
        \vspace{2mm}}
    }
    \end{center}
    % \markboth{#4 -- #1}{#4 -- #1}
    \vspace*{4mm}
}


\newcommand{\yahei}{\setCJKfamilyfont{yahei}{Microsoft YaHei} \CJKfamily{yahei}}




\newenvironment{solution}
{\color{blue} \paragraph{Solution.}}
{\newline \qed}

\begin{document}


%大标题在这里改!!!!!!!!!
\homework{实验报告}{2020 春}{陈家骏\phantom{ }黄书剑}{王晨渊}{181220057}

\lstset{numbers=left,numberstyle=\tiny,keywordstyle=\color{blue!70},commentstyle=\color{red!50!green!50!blue!50},frame=shadowbox, rulesepcolor=\color{red!20!green!20!blue!20},escapeinside=``,xleftmargin=2em,xrightmargin=2em, aboveskip=1em}

\section{概念题}
\subsection{C++中Lambda表达式的语法形式是怎样的?Lambda表达式有什么优点?}
用BNF可以写为

[<环境变量使用说明>] <形式参数> <返回值类型指定> <函数体>

优点在于可以将函数的定义与声明合二为一,同时减少了标识符的使用,为程序的编写带来便利。

\subsection{ C++中类成员访问运算符 -> 可以用于实现"智能指针"的功能。与正常指针相比,"智能指针"有哪些特点?}

    智能指针可以对堆区上的动态对象进行计数,当所有的指针都不指向某个动态对象时,这个对象可以被自动释放,从而使得程序员无需手动管理内存,大幅改善程序内存泄漏问题。

    此外,程序员可以自定义智能指针类进行一些其他操作。

\section{编程题}
\subsection{斐波那契数列}
\begin{lstlisting}[language={[ANSI]C++}]
    #include <cassert>
    class Fib
    {
    public:
        Fib(int index);
        Fib();
        int operator()();
    private:
        int last;
        int cur;
    
    };

    Fib::Fib(int index)
    {
        assert(index >= 1);
        last = 0;
        cur = 1;
        for (int i = 1; i < index; i++)
        {
            int temp = last + cur;
            last = cur;
            cur = temp;
        }
    }
    
    Fib::Fib()
    {
        last = 0;
        cur = 1;
    }
    
    int Fib::operator()()
    {
        int temp = last + cur;
        last = cur;
        cur = temp;
        return cur;
    }
 \end{lstlisting}
 \subsection{代码填空}
\begin{lstlisting}[language={[ANSI]C++}]
    struct Point
    {
        int x;
        int y;
    };
    
    bool IsTrue(Point &rstCenter, int iRadius, Point &P1, Point &P2)
    {
        auto PointInCircle = [rstCenter,iRadius](Point point)->bool
        {   
            return (point.x-rstCenter.x)*( point.x-rstCenter.x)+(point.y-rstCenter.y)*( point.y-rstCenter.y)<iRadius;
        };
        return PointInCircle(P1) != PointInCircle(P2);
    }    
\end{lstlisting}
\subsection{智能指针类}
    \begin{lstlisting}[language={[ANSI]C++}]
        class B
        {
        public:
            B(A* p);
            string string_access() const;
            A* operator->();
        private:
            string last_time;
            A* p_a;
        };
        
        B::B(A* p)
        {
            last_time = "No Access!";
            p_a = p;
        }
        
        string B::string_access() const
        {
            return last_time;
        }
        
        A* B::operator->()
        {
            time_t now_time=time(NULL);
            last_time =asctime(localtime(&now_time));
            return p_a;
        }
                
\end{lstlisting}

\end{document}
