\documentclass[UTF8]{ctexart}
%\documentclass[a4paper]{article}
% \usepackage[margin=1.25in]{geometry}
\usepackage[inner=2.0cm,outer=2.0cm,top=2.5cm,bottom=2.5cm]{geometry}
%\usepackage{CJK}
\usepackage{color}
\usepackage{graphicx}
\usepackage{amssymb}
\usepackage{amsmath}
\usepackage{amsthm}
\usepackage{bm}
\usepackage{hyperref}
\usepackage{multirow}
\usepackage{enumerate}
\usepackage{listings}
\usepackage{xcolor}
\usepackage{fontspec}
\setmainfont{Times New Roman}

\newcommand{\homework}[5]{
    \pagestyle{myheadings}
    \thispagestyle{plain}
    \newpage
    \setcounter{page}{1}
    \noindent
    \begin{center}
    \framebox{
        \vbox{\vspace{2mm}
        \hbox to 6.28in { {\bf 高级程序设计 \hfill #2} }
        \vspace{6mm}
        \hbox to 6.28in { {\Large \hfill \bf #1 \hfill} }
        \vspace{6mm}
        \hbox to 6.28in { {\it 指导老师: {\rm #3} \hfill 姓名: {\rm #4},学号: {\rm #5}}}
        \vspace{2mm}}
    }
    \end{center}
    % \markboth{#4 -- #1}{#4 -- #1}
    \vspace*{4mm}
}


\newcommand{\yahei}{\setCJKfamilyfont{yahei}{Microsoft YaHei} \CJKfamily{yahei}}




\newenvironment{solution}
{\color{blue} \paragraph{Solution.}}
{\newline \qed}

\begin{document}


%大标题在这里改!!!!!!!!!
\homework{实验报告}{2020 春}{陈家骏\phantom{ }黄书剑}{王晨渊}{181220057}

\lstset{numbers=left,numberstyle=\tiny,keywordstyle=\color{blue!70},commentstyle=\color{red!50!green!50!blue!50},frame=shadowbox, rulesepcolor=\color{red!20!green!20!blue!20},escapeinside=``,xleftmargin=2em,xrightmargin=2em, aboveskip=1em}

\section{概念题}
\subsection{简述C++中类属的概念。}
    一个程序实体能对多种类型的数据进行操作或描述的特性称为类属
\subsection{C++提供了哪两种实现类属函数的机制?简述它们的缺点。}
    \textbf{1}采用void* 类型的参数。
    比较麻烦,需要大量的指针操作。
    容易出错,编译程序无法进行类型检查。
    \textbf{2}函数模板。
    模板复用会产生相同的实例,需要额外处理。
\subsection{简述C++中参数化多态的概念及作用。}
    概念:一段带有类型参数的代码,给该参数提高不同类型就能得到多个不同的
    代码,即一段代码有多种解释。
    作用:实现源代码复用。

\section{编程题}
\subsection{Phone}
\begin{lstlisting}[language={[ANSI]C++}]
    #include<iostream>
    using namespace std;
    template <typename T>
    void f(T){ cout<<"f(T)"<<endl; }
    template <typename T>
    void f(const T*){ cout<<"f(const T*)"<< endl; }
    template <typename T>
    void g(T){ cout<<"g(T)"<<endl; }
    template <typename T>
    void g(T*){ cout<<"g(T*)"<<endl; }
    int main(){
        int a=1;
        int* b=&a;
        const int c=0;
        const int*d=&c;
        f(a);//f(T) int
        f(b);//f(T) int*
        f(c);//f(T) const int
        f(d);//f(const T*) int
        g(a);//g(T) int
        g(b);//g(T*) int
        g(c);//g(T) const int
        g(d);//g(T*) const int
    }
\end{lstlisting}
\subsection{链表}
\begin{lstlisting}[language={[ANSI]C++}]
    #include <iostream>
    #include <string>
    using namespace std;
    template <class T> 
    class Node
    {
    public:
        Node(){next=NULL;}
        T& val(){return value;}
        Node*& pnext(){return next;}
    private:
        T value;
        Node *next;
    };
    
    template<class T>
    class List
    {
    public:
        List()
        {
            head=NULL;
            tail=NULL;
        }
        ~List()
        {
            Node<T> *tmp =head;
            if(!tmp)
            {
                Node<T> *tmp2=tmp;
                tmp = tmp->pnext();
                delete tmp2;
            }
        }
        void add(T val)
        {
            if(head!=NULL)
            {
                tail->pnext() = new Node<T>;
                tail->pnext()->val()=val;
                tail = tail->pnext();
            }
            else
            {
                tail = new Node<T>;
                tail->val()=val;
                head=tail;   
            }
        }
        void display()
        {
            if(head!=NULL)
            {
                Node<T> *tmp = head;
                while(tmp!=NULL)
                {
                    cout<<tmp->val()<<" ";
                    tmp = tmp->pnext();
                }
            }
            
        }
    
    private:
        Node<T> *head,*tail;
    };
    
    int main()
    {
        List<int> mylist;
        mylist.add(21);
        mylist.add(12);
        mylist.display();
        cout<<endl;
        List<double> mylist2;
        mylist2.add(1.2);
        mylist2.add(2.1);
        mylist2.display();
        cout<<endl;
        List<string> mylist3;
        mylist3.add("abd");
        mylist3.add("adadada");
        mylist3.display();
    }
\end{lstlisting}
\subsection{矩阵}
\begin{lstlisting}[language={[ANSI]C++}]
    #include <iostream>
    #include <cassert>
    using namespace std;
    template <class T> class Matrix
    {
    public:
        Matrix(int m, int n)
        {
            row = m;
            col = n;
            all = new T*[m];
            for(int i=0;i<m;i++)
            {
                all[i] = new T[n];
            }
        }
        ~Matrix()
        {
            for(int i=0;i<row;i++)
            {
                delete[] all[i];
            }
            delete[] all;
        }
    
        Matrix(const Matrix& mat)
        {
            row = mat.row;
            col = mat.col;
            all = new T*[row];
            for(int i=0;i<row;i++)
            {
                all[i] = new T[col];
            }        
            for(int i=0;i<row;i++)
            {
                for(int j=0;j<col;j++)
                {
                    all[i][j]=mat.all[i][j];
                }
            }       
        }
        Matrix& operator= (const Matrix& mat)
        {
            assert(row==mat.row && col==mat.col);
            if(this==*mat) return *this;
            for(int i=0;i<row;i++)
            {
                for(int j=0;j<col;j++)
                {
                    all[i][j]=mat.all[i][j];
                }
            }
        }
        void display()
        {
            for(int i=0;i<row;i++)
            {
                for(int j=0;j<col;j++)
                {
                    cout<<all[i][j]<<" ";
                }
                cout<<endl;
            }        
        }
        void transpose()
        {
            T** new_all= new T*[col];
            for(int i=0;i<col;i++)
            {
                new_all[i] = new T[row];
            }
            for(int i=0;i<row;i++)
            {
                for(int j=0;j<col;j++)
                {
                    new_all[j][i]=all[i][j];
                }
            }                 
            for(int i=0;i<row;i++)
            {
                delete[] all[i];
            }
            delete[] all;
            all = new_all;
            int tmp = row;
            row = col;
            col = tmp;
        }
        void setMatrix()
        {
            for(int i=0;i<row;i++)
            {
                for(int j=0;j<col;j++)
                {
                    cin>>all[i][j];
                }
            }            
        }
        void debug()
        {
            cout<<"row:"<<row<<"col:"<<col<<endl;
        }
    
    
        Matrix operator+(const Matrix &mat)
        {
            Matrix res(*this);
            for(int i=0;i<row;i++)
            {
                for(int j=0;j<col;j++)
                {
                    res.all[i][j]=mat.all[i][j]+all[i][j];
                }
            }    
            return res;
        }
        Matrix operator*(const Matrix &mat)
        {
            assert(col==mat.row);
            Matrix res(row, mat.col);
            for(int i=0;i<row;i++)
            {
                for(int j=0;j<mat.col;j++)
                {
                    T sum;
                    for(int k=0;k<mat.row;k++)
                    {
                        if(!k)
                        {
                            sum = all[i][k]*mat.get(k,j);
                        }
                        else
                        {
                            sum+= all[i][k]*mat.get(k,j);
                        }
                    }
                    res.get(i,j)=sum;
                }
            }
            return res;
        }
        void square()
        {
            Matrix<T> copy(*this);
            copy.transpose();
            Matrix<T>res = *this * copy;
            res.display();
    
        }
        T& get(int i, int j)
        {
            assert(i<row&&j<col);
            return all[i][j];
        }
        T get(int i, int j) const
        {
            assert(i<row&&j<col);
            return all[i][j];        
        }
    
    
    
    private:
        T** all;
        int row,col;
    };
    
    
    
    int main()
    {
    #ifdef TEST1
        Matrix<int> matrix1(2,3);
        matrix1.setMatrix();
        matrix1.display();
        matrix1.transpose();
        matrix1.display();
        Matrix<int> matrix2(matrix1);
        matrix2.transpose();
        matrix2.display();
    
    #else
    
        Matrix<double> matrix1(2,3);
        matrix1.setMatrix();
        matrix1.display();
        Matrix<double> matrix2(2,2);
        matrix2.setMatrix();
        matrix2.square();
        Matrix<double> matrix3(2,3);
        matrix3.setMatrix();
        (matrix1+matrix3).display();
    #endif
    
    }
    
\end{lstlisting}

\end{document}
