\documentclass[UTF8]{ctexart}
%\documentclass[a4paper]{article}
% \usepackage[margin=1.25in]{geometry}
\usepackage[inner=2.0cm,outer=2.0cm,top=2.5cm,bottom=2.5cm]{geometry}
%\usepackage{CJK}
\usepackage{color}
\usepackage{graphicx}
\usepackage{amssymb}
\usepackage{amsmath}
\usepackage{amsthm}
\usepackage{bm}
\usepackage{hyperref}
\usepackage{multirow}
\usepackage{enumerate}
\usepackage{listings}
\usepackage{xcolor}
\usepackage{fontspec}
\setmainfont{Times New Roman}

\newcommand{\homework}[5]{
    \pagestyle{myheadings}
    \thispagestyle{plain}
    \newpage
    \setcounter{page}{1}
    \noindent
    \begin{center}
    \framebox{
        \vbox{\vspace{2mm}
        \hbox to 6.28in { {\bf 高级程序设计 \hfill #2} }
        \vspace{6mm}
        \hbox to 6.28in { {\Large \hfill \bf #1 \hfill} }
        \vspace{6mm}
        \hbox to 6.28in { {\it 指导老师: {\rm #3} \hfill 姓名: {\rm #4},学号: {\rm #5}}}
        \vspace{2mm}}
    }
    \end{center}
    % \markboth{#4 -- #1}{#4 -- #1}
    \vspace*{4mm}
}


\newcommand{\yahei}{\setCJKfamilyfont{yahei}{Microsoft YaHei} \CJKfamily{yahei}}




\newenvironment{solution}
{\color{blue} \paragraph{Solution.}}
{\newline \qed}

\begin{document}


%大标题在这里改!!!!!!!!!
\homework{实验报告}{2020 春}{陈家骏\phantom{ }黄书剑}{王晨渊}{181220057}

\lstset{numbers=left,numberstyle=\tiny,keywordstyle=\color{blue!70},commentstyle=\color{red!50!green!50!blue!50},frame=shadowbox, rulesepcolor=\color{red!20!green!20!blue!20},escapeinside=``,xleftmargin=2em,xrightmargin=2em, aboveskip=1em}

\section{概念题}
\subsection{请说出C++中同类对象共享数据的两种方式,并比较它们的优缺点。}
一种方法是使用全局变量存储共享数据。优点在于使用自由,书写简便。缺点在于缺少对数据的保护,由于其不受类的访问控制的限制,任何类的对象都可以自由的访问或修改。

另一种方法是使用静态数据成员和静态成员函数。优点是在实现数据共享的同时,一方面使其收到类的访问控制的限制,提高数据的安全性,另一方面静态数据成员对该类的所有对象仅存在一个拷贝,节省了内存使用。缺点在于使用时要遵循繁琐的访问控制规范,代码不够简洁。




\subsection{下面对静态数据成员的描述中,正确的是(B)}
A. 静态数据成员不可以通过类的对象调用

B. 静态数据成员可以在类体内进行初始化

C. 静态数据成员不能受private(私有)控制符的作用

D. 静态数据成员可以直接通过类名调用
\subsection{已知类A是类B的友元,类B是类C的友元,则(C)}
A. 类A一定是类C的友元

B. 类C一定是类A的友元

C. 类C的成员函数可以访问类B的对象的任何成员

D. 类A的成员函数可以访问类B的对象的任何成员
\subsection{简述C++中的迪米特法则(Law Of Demeter),遵循迪米特法则设计的模块具有哪些优点?}
    一个类的成员函数除了能访问自身类结构的直接子结构(表层子结构)外,不能以任何方式依赖于任何其他类的结构;并且每个成员函数只应向某个有限集合中的对象发送信息。
    
    对于类C中的任何成员函数M,M中能直接访问或引用的对象必须属于下述类之一:

    1)类C本身
    
    2)成员函数M的参数类

    3)M或M所调用的成员函数所创建的对象类

    4)全局对象所属的类
    
    5)类C的成员对象所属的类

    好处在于可以降低模块间的耦合度和成员函数对环境的依赖性。




\section{编程题}
\subsection{Array类和Matrix类}
\begin{lstlisting}[language={[ANSI]C++}]

    class Array
    {
    public:
        Array(int length, int column)
            : len(length), col(column)
        {
            array = new double[length];
        }
        Array(int length, int column, double origin[], int origin_len)
            : len(length), col(column)
        {
            array = new double[length];
            for (int i = 0; i < len; i++)
            {
                if (i < origin_len)
                {
                    array[i] = origin[i];
                }
                else
                {
                    array[i] = 0;
                }
            }
        }
        inline double loc(int x, int y) const
        {
            return array[x * col + y];
        }
        inline double get(int x) const
        {
            return array[x];
        }
    
        inline int get_col() const
        {
            return col;
        }
        inline int get_row() const
        {
            return len / col;
        }
        inline int get_len() const
        {
            return len;
        }
        inline void swap()
        {
            for (int i = 0; i < len; i++)
            {
                array[i] = -array[i];
            }
        }
        inline void output() const
        {
            for (int i = 0; i < len; i++)
            {
                if (i % col == col - 1)
                {
                    cout << array[i] << ",\n";
                }
                else
                {
                    cout << array[i] << ",";
                }
            }
        }
        Array(const Array &a)
        {
            len = a.get_len();
            col = a.get_col();
            array = new double[len];
            for (int i = 0; i < len; i++)
            {
                array[i] = a.get(i);
            }
        }
        ~Array()
        {
            delete[] array;
        }
    
    private:
        int len;
        int col;
        double *array;
    };
    
    class Matrix
    {
    public:
        Matrix(int this_row, int this_col, double origin[], int origin_len)
            : row(this_row),
              col(this_col),
              arr(row * col, col, origin, origin_len)
        {
        }
        Matrix(const Matrix &a)
            : row(a.get_row()),
              col(a.get_col()),
              arr(a.arr)
        {
            arr.swap();
        }
        inline int get_col() const
        {
            return col;
        }
        inline int get_row() const
        {
            return row;
        }
        inline void output() const
        {
            arr.output();
        }
    
    private:
        int row, col;
        Array arr;
    };    
 \end{lstlisting}
 \subsection{String类及其拼接函数}
\begin{lstlisting}[language={[ANSI]C++}]
    class String
    {
    public:
        String(char* s)
        {
            if(s)
            {
                len = strlen(s);
                str = new char[len+1];
                strcpy(str,s);
            }
            else
            {
                len = 0;
                str = NULL;
            }
        }
        String(const String& s)
        {
            len = s.get_len();
            set(s.get_str());
        }
        int get_len() const
        {
            return len;
        }
        char* get_str() const
        {
            return str;
        }
        bool is_NULL() const
        {
            return len==0;
        }
        
        void set(char* s)
        {
            if(s)
            {
                delete[] str;            
            }
            int s_len = strlen(s);
            str = new char[s_len+1];
            strcpy(str,s);
        }
        void clear()
        {
            delete[] str;
            str = NULL;
        }
    
        static int mystrcat(const String& str1, const String& str2, String& str3)
        {
            if(str1.is_NULL()&&str2.is_NULL())
            {
                cerr<<"Both are NULL\n";
                return 0;
            }
            else if(str1.is_NULL())
            {
                str3.set(str2.get_str());
                return strlen(str2.get_str());       
            }
            else if(str2.is_NULL())
            {
                str3.set(str1.get_str());
                return strlen(str1.get_str());  
            }
            else
            {
                int new_s_len = str1.get_len()+str2.get_len();
                char* new_s = new char[new_s_len+1];
                strcpy(new_s,str1.get_str());
                strcpy(&new_s[str1.get_len()],str2.get_str());
                str3.set(new_s);
                delete new_s;
                return strlen(new_s);          
            }
        }
        void output()
        {
            cout<<str<<endl;
        }
        ~String()
        {
            delete[] str;
            str = NULL;
            len = 0;
        }
    private:
        char* str;
        int len;
    };
\end{lstlisting}
\subsection{Single类}
    规定只能通过Single::creat\_obj()和Single::del\_obj()两个函数来创建或消亡Single类对象。
    \begin{lstlisting}[language={[ANSI]C++}]
        class Single
        {
        public:
            static Single* create_obj();
            static void del_obj();
            Single()
            {
                count_object++;
                
            }
            ~Single()
            {
                count_object--;
            }
        
        private:
            static int get_count_obj()
            {
                return count_object;
            }
            static int count_object;
            static Single* the_only;
        };
        int Single::count_object = 0;
        Single* Single::the_only = NULL;
        Single* Single::create_obj()
        {
            if(Single::get_count_obj())
            {
                cerr<<"Object Creation Failed!"<<endl;
            }
            else
            {
                Single::the_only = new Single;
                Single::count_object++;
                
            }
            return Single::the_only;
            
        }
        void Single::del_obj()
        {
            if(Single::get_count_obj())
            {
                Single::count_object--;
                delete Single::the_only;
                Single::the_only = NULL;
            }
            else
            {
                cerr<<"There is no object!"<<endl;
            }
        }
    \end{lstlisting}
\end{document}
