\documentclass[UTF8]{ctexart}
%\documentclass[a4paper]{article}
% \usepackage[margin=1.25in]{geometry}
\usepackage[inner=2.0cm,outer=2.0cm,top=2.5cm,bottom=2.5cm]{geometry}
%\usepackage{CJK}
\usepackage{color}
\usepackage{graphicx}
\usepackage{amssymb}
\usepackage{amsmath}
\usepackage{amsthm}
\usepackage{bm}
\usepackage{hyperref}
\usepackage{multirow}
\usepackage{enumerate}
\usepackage{listings}
\usepackage{xcolor}
\usepackage{fontspec}
\setmainfont{Times New Roman}

\newcommand{\homework}[5]{
    \pagestyle{myheadings}
    \thispagestyle{plain}
    \newpage
    \setcounter{page}{1}
    \noindent
    \begin{center}
    \framebox{
        \vbox{\vspace{2mm}
        \hbox to 6.28in { {\bf 高级程序设计 \hfill #2} }
        \vspace{6mm}
        \hbox to 6.28in { {\Large \hfill \bf #1 \hfill} }
        \vspace{6mm}
        \hbox to 6.28in { {\it 指导老师: {\rm #3} \hfill 姓名: {\rm #4},学号: {\rm #5}}}
        \vspace{2mm}}
    }
    \end{center}
    % \markboth{#4 -- #1}{#4 -- #1}
    \vspace*{4mm}
}


\newcommand{\yahei}{\setCJKfamilyfont{yahei}{Microsoft YaHei} \CJKfamily{yahei}}




\newenvironment{solution}
{\color{blue} \paragraph{Solution.}}
{\newline \qed}

\begin{document}


%大标题在这里改!!!!!!!!!
\homework{课程设计2}{2020 春}{陈家骏\phantom{ }黄书剑}{王晨渊}{181220057}

\lstset{numbers=left,numberstyle=\tiny,keywordstyle=\color{blue!70},commentstyle=\color{red!50!green!50!blue!50},frame=shadowbox, rulesepcolor=\color{red!20!green!20!blue!20},escapeinside=``,xleftmargin=2em,xrightmargin=2em, aboveskip=1em}
\section{编译运行环境}
    使用了VS2019进行编译运行。请将windows控制台放大到最大以改善使用体验。

\section{操作与规则}
\subsection{操作}
    仅实现了基本要求。玩家出生在屏幕下方。
    WASD操控移动,J发射炮弹。
\subsection{规则}
    每击杀一辆坦克积1分,
    敌人的内讧击杀同样积1分。
    玩家共两条生命。

    共3辆敌军坦克。
    轻型坦克生命为1,速度为3,为蓝色。中型坦克生命为2,速度为2,为红色。重型坦克生命为3,速度为1,为绿色。玩家为轻型坦克。

    每一点生命可以承受一发炮弹,生命规律后死亡。

    屏幕下发红色星星为基地,任意炮弹命中星星会导致游戏失败。

    胜利条件为所有敌人死亡。失败条件为基地被击中或玩家控制的坦克生命为0。

\section{设计思路}
\subsection{核心思路}
    本质上,程序中有一个画板,画板中的每个像素点都存储了对应的属性信息。所谓的tank 炮弹等等只是一个个管理员,用于管理对应的一些像素点。游戏的交互本质上是这些管理员的交互。
\subsection{代码结构}
    common中定义了项目中所有需要的宏和一些全局变量,此外包括了一些特殊函数的实现,如将输入映射到UP等宏的映射函数。

    Screen类型对象实现输入、交互、更新功能。敌人的行为利用随机数,复用Screen的输入模块实现。Screen的成员对象包括子弹管理类(Bullets),item*指针等等。

    Bullet, Tank都继承Item类型。Item是一个纯虚类。由于设计的失误,他们对应的方向等等信息接口实际上并不完全相容。为了保证编译通过,因此编写了一些冗余代码。

    Bullets是Bullet类型对象的管理员,将系统中可能的所有bullet类型对象都设置为其成员对象。

    Vector2D类是坐标类型的一个包装,直接复用了上次贪吃蛇的设计。

    Block是像素点,保存了单个像素的信息,如颜色、类型等。自认为一个比较脏的设计是,像素的信息中,包含了一个指针,用于指向管理这个像素的管理员。这样明显破坏了数据的封装,但是实际上带来了极大的便捷,可以迅速索引到对应的管理员从而完成一些更新。

    LightTank,MediumTank,HeavyTank仅仅是对Tank的一些成员的信息进行了修改,如修改速度、血量。虽然继承自Tank,但是实际上没有本质区别。


    游戏的进行采取了传统的批处理系统的模式,以Sreen为核心进行。print\_text和print\_graphic是真实的输出函数,用于输出图像和游戏的统计信息。

\section{尚未解决的问题}
\subsection{虚基类设置不合理}
    Item作为虚基类,其两个派生类bullet和tank需要的接口明显不同,同时程序中所有的指针都是Item*,直接导致了程序中塞了很多的无意义代码用来保证编译通过。
    解决办法是彻底重构调整结构,同样由于时间问题没有进行。
\subsection{"画板"设置不合理}
    画板数组应当直接设置为全局变量,而非私有成员。设置不合理导致程序中存储了大量完全一样的画板指针,造成了极大的信息冗余。由于时间来不及,没有进行优化。

\subsection{刷新速率过慢,有明显屏闪}
    我通过关闭cout的与stdio的协同来加快速度,但是很遗憾问题没有完全解决。我认为可能需要一些移动光标的库函数,仅刷新改变的部分才有效。但是由于时间有限没有更新。    

\end{document}