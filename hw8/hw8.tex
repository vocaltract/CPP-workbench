\documentclass[UTF8]{ctexart}
%\documentclass[a4paper]{article}
% \usepackage[margin=1.25in]{geometry}
\usepackage[inner=2.0cm,outer=2.0cm,top=2.5cm,bottom=2.5cm]{geometry}
%\usepackage{CJK}
\usepackage{color}
\usepackage{graphicx}
\usepackage{amssymb}
\usepackage{amsmath}
\usepackage{amsthm}
\usepackage{bm}
\usepackage{hyperref}
\usepackage{multirow}
\usepackage{enumerate}
\usepackage{listings}
\usepackage{xcolor}
\usepackage{fontspec}
\setmainfont{Times New Roman}

\newcommand{\homework}[5]{
    \pagestyle{myheadings}
    \thispagestyle{plain}
    \newpage
    \setcounter{page}{1}
    \noindent
    \begin{center}
    \framebox{
        \vbox{\vspace{2mm}
        \hbox to 6.28in { {\bf 高级程序设计 \hfill #2} }
        \vspace{6mm}
        \hbox to 6.28in { {\Large \hfill \bf #1 \hfill} }
        \vspace{6mm}
        \hbox to 6.28in { {\it 指导老师: {\rm #3} \hfill 姓名: {\rm #4},学号: {\rm #5}}}
        \vspace{2mm}}
    }
    \end{center}
    % \markboth{#4 -- #1}{#4 -- #1}
    \vspace*{4mm}
}


\newcommand{\yahei}{\setCJKfamilyfont{yahei}{Microsoft YaHei} \CJKfamily{yahei}}




\newenvironment{solution}
{\color{blue} \paragraph{Solution.}}
{\newline \qed}

\begin{document}


%大标题在这里改!!!!!!!!!
\homework{实验报告}{2020 春}{陈家骏\phantom{ }黄书剑}{王晨渊}{181220057}

\lstset{numbers=left,numberstyle=\tiny,keywordstyle=\color{blue!70},commentstyle=\color{red!50!green!50!blue!50},frame=shadowbox, rulesepcolor=\color{red!20!green!20!blue!20},escapeinside=``,xleftmargin=2em,xrightmargin=2em, aboveskip=1em}

\section{概念题}
\subsection{分别简述C++中单继承和多继承的概念}
    单继承指一个派生类只有一个直接基类。

    多继承指一个派生类可以有两个或两个以上的直接基类。
\subsection{说明C++代码复用中,继承和聚集这两种方式的区别?}
    继承与封装存在矛盾,但集聚则没有。
    继承可以实现子类型,但是具有集聚关系的两个类不具备子类型关系。
\subsection{简述C++中虚基类的作用}
    解决重复继承问题带来的空间浪费与冗余的一致性检查。


\section{编程题}
\subsection{阅读代码}
    我预测的输出结果是

    string1

    string2

    string3

    string4

    destructor of D

    destructor of C

    destructor of B

    destructor of A
    
    实际输出结果与预测一致。
\subsection{Phone}
\begin{lstlisting}[language={[ANSI]C++}]
    #include <iostream>
    #include <cstdio>
    #include <Windows.h>
    #include <cstring>
    #include<time.h>
    using std::cout;
    using std::endl;
    using std::cin;
    #pragma warning(disable:4996)
    class Screen
    {
    public:
        void display(char* message, int len);
        Screen(int l, int w);
    private:
        int length;
        int width;
    };
    int code[200];
    
    class Mainboard
    {
    public:
        Mainboard();
        Mainboard(int d);
        void encode(char* message, int* code, int len);
        void decode(char* message, int* code, int len);
        int decoding(char s);
        int encoding(char s);
    private:
        int delay;
    };
    
    class TestPhone
    {
    public:
        TestPhone();
        void sendMessage();
        void sendMessage(char* message);
        void receiveMessage();
        void inputAndDisplay();
        void inputAndDisplay(char* message);
    protected:
        Mainboard mainboard;
        Screen screen;
        int code_len;
    };
    
    class ReleasePhone:public TestPhone
    {
    public:
        void dateAndTime();
    };
    
    Screen::Screen(int l, int w)
    {
        length = l;
        width = w;
    }
    
    void Screen::display(char* message, int len)
    {
        len--;
        char* s = new char[width+1];
        s[width] = '\0';
        int outp_cont = len / width * width == len ? len / width : len / width + 1;
        for (int i = 0; i < outp_cont; i++)
        {
            if (i % length == 0 && i >= length)
            {
                printf("\n");
            }
            if (i < outp_cont - 1)
            {
                strncpy(s, &message[i * width], width);
                printf("%s\n", s);
            }
            else
            {
                strncpy(s, &message[i * width], len-i*width);
                s[len - i * width] = '\0';
                printf("%s", s);
            }
        }
    }
    
    Mainboard::Mainboard()
    {
        delay = 0;
    }
    Mainboard::Mainboard(int d)
    {
        delay = d;
    }
    int Mainboard::encoding(char s)
    {
        if (s >= 'a' && s <= 'z')
        {
            return s - 'a';
        }
        if (s == ' ')
        {
            return 26;
        }
        if (s == '\0')
        {
            return -1;
        }
        return s;
    }
    
    int Mainboard::decoding(char s)
    {
        if (s >= 0 && s <= 25)
        {
            return s + 'a';
        }
        if (s == 26)
        {
            return ' ';
        }
        if (s == -1)
        {
            return '\0';
        }
        return s;
    }
    
    void Mainboard::encode(char* message, int* code, int len)
    {
        for (int i = 0;i < len; i++)
        {
            code[i] = encoding(message[i]);
            Sleep(delay);
        }
    }
    
    void Mainboard::decode(char* message, int* code, int len)
    {
        for (int i = 0; i < len; i++)
        {
            message[i] = decoding(code[i]);
            Sleep(delay);
        }
    }
    
    TestPhone::TestPhone()
        :mainboard(1), screen(8,12)
    {
        code_len = 0;
    }
    
    void TestPhone::sendMessage()
    {
        char s[200];
        cin.get(s, 200);
        code_len = strlen(s)+1;
        mainboard.encode(s, code, code_len);
    }
    
    void TestPhone::receiveMessage()
    {
        char s[200];
        mainboard.decode(s, code, code_len);
        screen.display(s, code_len);
    }
    
    void TestPhone::inputAndDisplay()
    {
        sendMessage();
        receiveMessage();
    }
    
    void TestPhone::sendMessage(char* message)
    {
        code_len = strlen(message) + 1;
        mainboard.encode(message, code, code_len);
    }
    
    void TestPhone::inputAndDisplay(char* message)
    {
        sendMessage(message);
        receiveMessage();
    }
    
    void ReleasePhone::dateAndTime()
    {
        char system_time[200];
        time_t now_time = time(NULL);
        strcpy(system_time, asctime(localtime(&now_time)));
        system_time[strlen(system_time) - 1] = '\0';
        inputAndDisplay(system_time);
    }
\end{lstlisting}

\end{document}
