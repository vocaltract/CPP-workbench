\documentclass[UTF8]{ctexart}
%\documentclass[a4paper]{article}
% \usepackage[margin=1.25in]{geometry}
\usepackage[inner=2.0cm,outer=2.0cm,top=2.5cm,bottom=2.5cm]{geometry}
%\usepackage{CJK}
\usepackage{color}
\usepackage{graphicx}
\usepackage{amssymb}
\usepackage{amsmath}
\usepackage{amsthm}
\usepackage{bm}
\usepackage{hyperref}
\usepackage{multirow}
\usepackage{enumerate}
\usepackage{listings}
\usepackage{xcolor}
\usepackage{fontspec}
\setmainfont{Times New Roman}

\newcommand{\homework}[5]{
    \pagestyle{myheadings}
    \thispagestyle{plain}
    \newpage
    \setcounter{page}{1}
    \noindent
    \begin{center}
    \framebox{
        \vbox{\vspace{2mm}
        \hbox to 6.28in { {\bf 高级程序设计 \hfill #2} }
        \vspace{6mm}
        \hbox to 6.28in { {\Large \hfill \bf #1 \hfill} }
        \vspace{6mm}
        \hbox to 6.28in { {\it 指导老师: {\rm #3} \hfill 姓名: {\rm #4},学号: {\rm #5}}}
        \vspace{2mm}}
    }
    \end{center}
    % \markboth{#4 -- #1}{#4 -- #1}
    \vspace*{4mm}
}


\newcommand{\yahei}{\setCJKfamilyfont{yahei}{Microsoft YaHei} \CJKfamily{yahei}}




\newenvironment{solution}
{\color{blue} \paragraph{Solution.}}
{\newline \qed}

\begin{document}


%大标题在这里改!!!!!!!!!
\homework{实验报告}{2020 春}{陈家骏\phantom{ }黄书剑}{王晨渊}{181220057}

\lstset{numbers=left,numberstyle=\tiny,keywordstyle=\color{blue!70},commentstyle=\color{red!50!green!50!blue!50},frame=shadowbox, rulesepcolor=\color{red!20!green!20!blue!20},escapeinside=``,xleftmargin=2em,xrightmargin=2em, aboveskip=1em}

\section{概念题}
\subsection{C++中输入/输出(I/O)分成几类?分别是什么?}
\subsubsection{面向控制台的I/O}
    从标准输入设备中获得数据,把程序结果从标准输出设备中输出。
\subsubsection{面向文件的I/O}
    从外存文件中获得数据,把程序结果保存到外存文件中。
\subsubsection{面向字符串变量的I/O}
    从程序中的字符串变量中获取数据,把程序结果保存到字符串变量中。
\subsection{请简述C++中流式文件的概念。}
    从手册上抄来的。
A stream is an abstraction that represents a device on which input and ouput operations are performed. A stream can basically be represented as a source or destination of characters of indefinite length.
\subsection{请简述C++中对文件数据进行读写的过程。}
    创建对应的fstream类,保证打开成功后进行读写,读写完成后调用close方法。

\section{编程题}
\subsection{表格}
\begin{lstlisting}[language={[ANSI]C++}]
    #include <iostream>
    #include <iomanip>
    #include <cmath>
    using namespace std;
    #define PI 3.141592653589
    int main()
    {
        cout.setf(ios::left);
        cout<<setw(3)<<"x";
        cout<<setw(10)<<"sin(x)"<<setw(10)<<"cos(x)"<<setw(10)<<"tan(x)"<<endl;
        for(int i=1;i<=10;i++)
        {
            cout<<setw(3)<<i;
            cout<<setw(10)<<setiosflags(ios::fixed)<<setprecision(5)<<sin((double)i/180*PI);
            cout<<setw(10)<<setiosflags(ios::fixed)<<setprecision(5)<<cos((double)i/180*PI); 
            cout<<setw(10)<<setiosflags(ios::fixed)<<setprecision(5)<<tan((double)i/180*PI);              
            cout<<endl;
        }
    }
\end{lstlisting}
\subsection{时间分配}
\begin{lstlisting}[language={[ANSI]C++}]
    #include <iostream>
    #include <fstream>
    #include <string>
    #include <cassert>
    #include <regex>
    #include <cstdlib>
    #include <vector>
    #include <algorithm>
    using std::ofstream;
    using std::string;
    using std::vector;
    using std::ios;
    using std::cout;
    using std::endl;
    using std::ifstream;
    using std::cin;
    using std::regex;
    using std::smatch;
    using std::stoi;
    
    int main()
    {
        ofstream fileA("A.dat",ios::out|ios::binary);
        ofstream fileB("B.dat",ios::out|ios::binary);
        if(!fileA.is_open()||!fileB.is_open()) assert(0);
        string str_buf;
        cout<<"Type the A's available workhours, and end with 'A'."<<endl;
        getline(cin,str_buf,'A');
        fileA.write(&str_buf[0],str_buf.length());
        fileA.close();
        str_buf.clear();
        cin.ignore();
        cout<<"Type the B's available workhours, and end with 'B'."<<endl;
        getline(cin,str_buf,'B');
        fileB.write(&str_buf[0],str_buf.length());
        fileB.close();
        ifstream in_fileA("A.dat",ios::in|ios::binary);
        ifstream in_fileB("B.dat",ios::in|ios::binary);
        if(!in_fileA.is_open()||!in_fileB.is_open()) assert(0);
        str_buf.resize(200);
        in_fileA.read(&str_buf[0],200);
        regex reg("([0-9]{1,2}):00~([0-9]{1,2}):00");
        smatch sm;
        vector<int> a_times,b_times;
        while(regex_search(str_buf,sm,reg))
        {
            for(int i=stoi(sm[1]);i<stoi(sm[2]);i++)    
                a_times.push_back(i);
            str_buf = sm.suffix().str();
        }
        str_buf.clear();
        str_buf.resize(200);
        in_fileB.read(&str_buf[0],200);
        while(regex_search(str_buf,sm,reg))
        {
            for(int i=stoi(sm[1]);i<stoi(sm[2]);i++)    
                b_times.push_back(i);
            str_buf = sm.suffix().str();
        }
        vector<int> res(25);
        vector<int>::iterator it=std::set_intersection(a_times.begin(),a_times.end(),b_times.begin(),b_times.end(),res.begin());
        res.resize(it-res.begin());
        bool first_flag=true;
        int last_val=0;
        std::ostringstream res_buf;
        for(auto it=res.begin();it!=res.end();it++)
        {
            if(first_flag)
            {
                first_flag=false;
                res_buf<<*it<<":00~";
                last_val=*it;
            }
            else if(last_val+1!=*it)
            {
                res_buf<<last_val+1<<":00,";
                res_buf<<*it<<":00~";
                last_val=*it;
            }
            else
            {
                last_val++;
            }
        }
        res_buf<<last_val+1<<":00";
        ofstream fileC("C.dat",ios::out|ios::binary);
        fileC.write(&(res_buf.str())[0],res_buf.str().length());
        fileC.close();
    }
\end{lstlisting}
\subsection{员工管理}
\begin{lstlisting}[language={[ANSI]C++}]
    #include <iostream>
    #include <vector>
    #include <string>
    #include <cassert>
    #include <fstream>
    #include <regex>
    #include <iomanip>
    using std::string;
    using std::vector;
    using std::ios;
    using std::cout;
    using std::endl;
    using std::cin;
    using std::regex;
    using std::smatch;
    using std::setw;
    #define BACKOFFSET (-(sizeof("ID:    ,Name:         ,Phone:         ,Post ID:        ,Address:              \n")-1))
    class Worker
    {
    public:
        Worker(string ID,string name, string phone,string post,string address);
        string& get_id();
    private:
        string _ID;
        string _name;
        string _phone;
        string _post;
        string _address;
    friend std::ostream& operator <<(std::ostream& out, const Worker& x);
    };
    std::ostream& operator <<(std::ostream& out, const Worker& x)
    {
        out.setf(ios::left);
        out<<"ID:"<<setw(4)<<x._ID;
        out<<",Name:"<<setw(9)<<x._name;
        out<<",Phone:"<<setw(9)<<x._phone;
        out<<",Post ID:"<<setw(8)<<x._post;
        out<<",Address:"<<setw(14)<<x._address;
        return out;
    }
    
    class AddressBook
    {
    public:
        AddressBook();
        ~AddressBook();
        void add(Worker &worker);
        Worker search(string id);
        void modify(Worker &worker);
    
    private:
        std::fstream file;
        regex reg;
    };
    
    
    
    int main()
    {
        Worker list[]=
        {
            {"1","Wang","150","123","Shanghai"},
            {"2","Li","139","123","Shanghai"},
            {"3","Zhao","110","321","Beijing"},
            {"4","Qian","189","101","Nanjing"},
            {"5","Sun","150","123","Shanghai"},                   
        };
        AddressBook book;
        for(int i=0;i<5;i++)
        {
            book.add(list[i]);
        }
        cout<<book.search("1")<<endl;
        Worker wgai("5","Liu","150","123","Shandong");
        book.modify(wgai);
        Worker wgai2("2","Kai","150","123","Tianjing");
        book.modify(wgai2);
        cout<<book.search("2")<<endl;
        cout<<book.search("10")<<endl;
    }
    
    Worker::Worker(string ID,string name, string phone,string post,string address)
        :_ID(ID),_name(name),_phone(phone),_post(post),_address(address){}
    
    string& Worker::get_id()
    {
        return _ID;
    }
    
    AddressBook::AddressBook()
        :reg("ID:([0-9]+) {0,10},Name:([A-Za-z]+) {0,10},Phone:([0-9]+) {0,10},Post ID:([0-9]+) {0,10},Address:([A-Za-z]+) {0,10}")
    { 
       file.open("workers.txt",ios::out|ios::in);
       if(!file.is_open())
       {
            file.open("workers.txt",ios::out);
            assert(file.is_open());
            file.close();
            file.open("workers.txt",ios::out|ios::in);
            assert(file.is_open());
       }
    }
    
    AddressBook::~AddressBook()
    {
        assert(file.is_open());
        file.close();
    }
    
    void AddressBook::add(Worker &worker)
    {
        file<<worker<<endl;
    }
    
    Worker AddressBook::search(string id)
    {
        string buf;
        file.seekg(ios::beg);
        getline(file,buf);
        smatch sm;
        while(!file.eof())
        {
            if(regex_search(buf,sm,reg))
            {
                if(sm[1]==id)
                {
                    return (Worker){sm[1],sm[2],sm[3],sm[4],sm[5]};
                }
            }
            buf.clear();
            getline(file,buf);
        }
        file.seekg(ios::end);
        cout<<"No this worker!"<<endl;
        return (Worker){"","","","",""};
    }
    
    void AddressBook::modify(Worker &worker)
    {
        string buf;
        file.seekg(ios::beg);
        getline(file,buf);
        smatch sm;
        while(!file.eof())
        {
            if(regex_search(buf,sm,reg))
            {
                if(sm[1]==worker.get_id())
                {
                    file.seekg(BACKOFFSET,ios::cur);
                    file<<worker;
                    file.seekg(ios::end);
                    return;
                }
            }
            buf.clear();
            getline(file,buf);
        }
        file.seekg(ios::end);
        cout<<"No this worker!"<<endl;    
    }
\end{lstlisting}

\end{document}
